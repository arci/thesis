%--------------------------------------------------------------------------------
% abstract in italian
%--------------------------------------------------------------------------------
\thispagestyle{empty}

\chapter*{Estratto}
Negli ultimi anni, specialmente per le applicazioni web, i requisiti sulla gestione dei dati sono drasticamente cambiati; le applicazioni devono gestire dati che per natura non sono strutturati e, specialmente, vengono generati in una quantità tale che i sistemi tradizionali per la gestione dei dati non sono più convenienti. Varie soluzioni sonos tate proposte come alternative ai classici sistemi DBMS; queste soluzioni prendono il nome di database NoSQL, per sottolineare il differente approccio adottato rispetto ai tradizionali DBMS.

\noindent Molte soluzioni NoSQL sono apparse in questi anni, e ognuna di esse utilizza un approccio differente nel cercare di soddisfare i requisiti elencati sopra. Queste tecnologie forniscono un insisme di API proprietarie che, in generale, non astraggono dalla struttra fisica dei dati, questo mette l'utente in condizioni di dover scrivere molto più codice rispetto all'utilizzo di DBMS.

\noindent La mancanza di un linguaggio comune a tutti i database NSQL richiede, quando si sviluppa un applicazione, una chiara e precisa conoscenza delle soluzioni disponibili sul mercato, per essere in grado di scegliere la tecnologia che più soddisfa i reauisiti dell'applicazione; molti cambiamenti però, possono sorgere durante il ciclo di vita dell'applicazione e, cambiare la soluzione NoSQL adottata, può essere un problema. Questo problema è noto come \textit{vendor lock-in}.

\noindent Questo lavoro propone a modello che, usando la libreria CPIM e grazie alla ben nota interfaccia JPA , permetta all'utente di sviluppare applicazioni usando l'SQL come linguaggio comune per molte tecnologie NoSQL e ottenere così una buona protabilità del codice, mitigando la complessità delle tecnologie NoSQL senza perderne i vantaggi in termini di salabilità e performance. Inoltre proponiamo l'integrazione, all'interno della libreria CPI, del sistema di migrazione e sincronizzazione \textit{Hegira}, er gestire migrazione dei dati fra database NoSQL mitigando cosu il problema del vendor lock-in.

\cleardoublepage

%--------------------------------------------------------------------------------
% abstract in english
%--------------------------------------------------------------------------------
\thispagestyle{empty}

\chapter*{Abstract}
Within the last years, especially for web applications, data requirements have changed drastically; applications needs to handle information that are not always well structured and, more importantly, came in a volume that is not sustainable for traditional data management techniques. Solutions that tries to handle those new kind of data have emerged over the classical DBMS solutions; those solutions comes under the name of NoSQL databases, to underline the different approach they bring with respect to traditional  DBMS.

\noindent Many NoSQL technologies comes into play in these years and each of them uses a different approach to handle all of the above requirements. Those technologies provides a set of proprietary API that does not abstract the physical structure of the stored data, this move toward the user a lot of programming effort with respect to DBMS. 

\noindent The lack of a common language for NoSQL databases, require, when developing application, a clear understanding of the available NoSQL solutions, to be able to choose the right technology for the application requirements but, during the life cycle of the applications, many changes can arise but at this point, change the adopted NoSQL technology may become a problem. This problem is known as \textit{vendor lock-in}.  

\noindent This work proposes a model that, using the CPIM library and through the well-known JPA interface, permits to the users to develop applications using SQL as a common language to many NoSQL technologies and thus achieve code probability, leveraging the complexity of NoSQL systems while exploiting the advantages that those technologies bring in terms of scalability and performance. Moreover we propose the integration in the CPIM library of \textit{Hegira} a migration and synchronization system to handle data migration among NoSQL database mitigating thus the vendor lock-in problem.

\cleardoublepage