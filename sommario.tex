Se la tesi \`e  scritta in Inglese, deve essere incluso un sommario introduttivo in Italiano. Nel sommario devono essere riassunti tutti i contenuti della tesi. Di fatto, il sommario in Italiano pu\`o essere la semplice traduzione del primo capitolo introduttivo della tesi, di cui mantiene l'eventuale struttura a sezioni.

\section*{Prima sezione}
\dots

\section*{Seconda sezione}
\dots

\section*{Organizzazione}
Questa tesi \`e organizzata come segue. 
\begin{itemize}
\item Nel Capitolo \ref{chap:one} \dots
\item Nel Capitolo \ref{chap:two} \dots
\item Nel Capitolo \ref{chap:three} \dots
\item \dots
\end{itemize}
La tesi si conclude con il Capitolo \ref{chap:conclusions}. Sono
discussi i risultati ottenuti e sono presentati alcuni possibili
sviluppi futuri.


\section*{Contributi}
Questo lavoro si distingue per i seguenti contributi originali:
\begin{itemize}
\item \dots riassunto sintetico dei diversi contributi
\item \dots
\item \dots
\end{itemize}
