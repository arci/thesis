\section{Introduction}
In this chapter NoSQL databases are firstly introduced and compared with SQL solutions. In section \ref{sec:common-language} are listed some of the solution that has been developed in defining a common language or interface to interact with different NoSQL databases.
\noindent Finally the CPIM library is introduced as a tentative in defining a common interface to interacts with different vendor in a PaaS environment which include accessing the different NoSQL solution of the PaaS provider.

\section{NoSQL databases}
\subsection{NoSQL motivations}
\subsection{NoSQL characteristics}
\subsection{Standard language}

\section{Approaches for a common language}
\label{sec:common-language}
\subsection{Kundera}
\subsection{Spring-data}
\subsection{Espresso Logic}
\subsection{Couchbase UnQL}
\subsection{PlayORM}
\subsection{? Apache Gora} 
\subsection{? Apache Phoenix}
\subsection{? SOS Platform}

\section{Cloud Platform Independent Model}

\section{Summary}
In this chapter has been introduced some of the main reasons that leads to the NoSQL database definition and why industry is so interested in those kind of technology. 
\noindent NoSQL technology has been compared with SQL systems to highlight the deep differences but especially the lack of a common language definition. Have been discussed some of the major project that try to define a common language among different NoSQL solution.
\noindent Finally has been presented the CPIM library, a more general approach in a common language definition in PaaS environment.