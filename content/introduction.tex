Introduzione al lavoro. Inizia direttamente, senza nessuna sezione.

\noindent Argomenti trattati suddivisi sezione per sezione\dots

\noindent Per citare un articolo, ad esempio \cite{book:projpa2} o \cite{online:azuretable, online:datastore} utilizzare il comando \texttt{\\cite}. 

\noindent Per gestire i file di tipo \texttt{bib} esiste il programma \texttt{JabRef} disponibile sul sito \texttt{http://jabref.sourceforge.net/}.

\noindent Per includere degli algoritmi come l'Algoritmo~\ref{alg:esempio}
usare lo stile \texttt{algpseudocode} presente nel package~\texttt{algorithmicx}.

\begin{algorithm}[h]
  %%%
%%%
%%%	the Q-learning algorithm
%%%
%%%
\begin{algorithmic}[1]
\State Initialize $Q(\cdot,\cdot)$ arbitrarily
\ForAll{episodes}
   \State $t$ $\leftarrow$ 0
   \State Initialize $s_{t}$
   \Repeat
      \State $a_{t} \gets \pi(s_{t})$
      \State perform action $a_{t}$; observe $r_{t+1}$ and $s_{t+1}$
      \State $Q(s_{t},a_{t}) \gets Q(s_t,a_t) + \alpha( r_{t+1} + \gamma \max_{a\in A} Q(s_{t+1},a) -  Q(s_t,a_t))$
      \State $t \gets t+1$
   \Until{$s_{t}$ is terminal}
\EndFor
\end{algorithmic}

  \caption{Un esempio di algoritmo.}
  \label{alg:esempio}
\end{algorithm}

\section*{Original Contributions}
This work include the following original contributions:
\begin{itemize}
\item \dots riassunto sintetico dei diversi contributi
\item \dots
\item \dots
\end{itemize}

\section*{Outline of the Thesis}
This thesis is organized as follows: 
\begin{itemize}
\item In Chapter~\ref{chap:one} \dots
\item In Chapter~\ref{chap:two} \dots
\item In Chapter~\ref{chap:three} \dots
\item \dots
\end{itemize}
Finally, in Chapter~\ref{chap:conclusions}, \dots


