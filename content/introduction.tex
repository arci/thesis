In the last few years, due to the advent of Web 2.0, more and more data becomes available, generated by a growing multitude of people. The nature of this kind of data is intrinsically unstructured and comes in a volume that traditional data management techniques are no more affordable to guarantee modern application requirements.
In this scenario, NoSQL databases have emerged over traditional DBMS as a more suitable alternative to handle those new kinds of data.

\noindent NoSQL databases try to address the new applications requirements in terms of: fault tolerance, availability across distributed data sources, scalability and consistency, in different ways, proposing different properties and characteristics. 
Each NoSQL database thus provides to its users a different API interface tailored to exploit the specific characteristic that the NoSQL offer.

\noindent The lack of a common language for NoSQL databases, require, when developing an application, a clear understanding of the available NoSQL solutions, to be able to choose the right technology for the application requirements. However, during the life cycle of the application, changing the adopted NoSQL technology, maybe due to requirements changes, may become a problem. This problem is known as \textit{vendor lock-in}.  

\noindent To mitigate the problem, this work proposes an extension of the CPIM library, which already try to address the vendor lock-in problem in PaaS environments, in order to provide to the user a way to abstract from the specific NoSQL technology used to store the data.
Many solutions has been proposed by both communities and industry, in defining a common way to access different NoSQL technologies. We propose to use the one that seems to get most interest, especially by the industry, which is the use of the JPA interface. With this objective we integrate Kundera, an ORM for NoSQL databases based on the JPA standard interface in the NoSQL service of the CPIM library, furthermore we develop new extensions for Kundera to support the database currently supported by CPIM.
The complexity of using NoSQL databases is moved inside Kundera clients and this permit to the user to access several NoSQL technologies by using both a standard language, which is JPQL (a SQL-like language), and a well known interface among Java developer, the JPA interface.

\noindent To achieve complete portability, both of the code and of the stored data, this work also propose the integration, in the CPIM library, of the migration and synchronization system \textit{Hegira}. This permit to the user a way of moving its data from a NoSQL technology to another without experiencing application down time, and thus be able to change the NoSQL technology without the need of re-engineering the application.
 
\section*{Original Contributions}
This work include the following original contributions:
\begin{itemize}
\item two brand new Kundera clients, one for Google Datastore and one for Azure Tables;
\item the support, for the  NoSQL service of CPIM library, for the interaction with the migration system \textit{Hegira}, both for data synchronization and migration; 
\end{itemize}

\section*{Outline of the Thesis}
This thesis is organized as follows: 
\begin{itemize}
\item In Chapter \ref{chap:sota} is described the evolution of NoSQL. As a first introduction is discussed why in this years this technology have emerged over SQL solutions and what are the main differences among those technology, the second part aims to underline the lack of a common language for NoSQLs in contrast to DBMS in which SQL exists.
\item In Chapter \ref{chap:ps} we analyze the current available technologies for NoSQL databases and propose a work for extending the CPIM library to make its interaction with NoSQL world even easier and much more extensible. Furthermore we analyze the requirements of modern web application motivating the choice of integrating in the CPIM NoSQL service a mechanism of data migration.
\item Chapter \ref{chap:kundera} is dedicated to the develop of the two Kundera client extension that have been developed in order to support Google Datastore and Azure Tables that will be then used in CPIM as adapters for the relative database.
\item In Chapter \ref{chap:cpim} is presented the work made on CPIM. As a first step is described a modification in the CPIM NoSQL service aimed to integrate Kundera as unique persistence layer for NoSQL access using the standard JPA interface, the library was previously using several different JPA implementation one for each of the supported databases. Furthermore is discussed the extension of CPIM to include an interaction with the migration system \textit{Hegira}.
\item In Chapter \ref{chap:eval} are described the various tests that have been performed on the developed Kundera extensions, to guarantee the correctness of the operation and to provide a measurement of performance. Moreover is presented a test application developed to be able to test the migration of the data generated through the CPIM library through the NoSQL service.
\item In Chapter \ref{chap:conclusions} draws the conclusions on the entire work and proposes some possible future works.
\end{itemize}

